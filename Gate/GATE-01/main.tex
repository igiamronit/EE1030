%iffalse
\let\negmedspace\undefined
\let\negthickspace\undefined
\documentclass[journal,12pt,onecolumn]{IEEEtran}
\usepackage{cite}
\usepackage{amsmath,amssymb,amsfonts,amsthm}
\usepackage{algorithmic}
\usepackage{graphicx}
\usepackage{textcomp}
\usepackage{xcolor}
\usepackage{txfonts}
\usepackage{listings}
\usepackage{enumitem}
\usepackage{mathtools}
\usepackage{gensymb}
\usepackage{comment}
\usepackage[breaklinks=true]{hyperref}
\usepackage{tkz-euclide} 
\usepackage{listings}
\usepackage{gvv}                                        
%\def\inputGnumericTable{}                                 
\usepackage[latin1]{inputenc}                                
\usepackage{color}                                            
\usepackage{array}                                            
\usepackage{longtable}                                       
\usepackage{calc}                                             
\usepackage{multirow}                                         
\usepackage{hhline} 
\usepackage{ifthen}                                           
\usepackage{lscape}
\usepackage{tabularx}
\usepackage{array}
\usepackage{float}
\usepackage{multicol}
\usepackage{tikz}
\usetikzlibrary{graphs}
\usepackage{circuitikz}



\newtheorem{theorem}{Theorem}[section]
\newtheorem{problem}{Problem}
\newtheorem{proposition}{Proposition}[section]
\newtheorem{lemma}{Lemma}[section]
\newtheorem{corollary}[theorem]{Corollary}
\newtheorem{example}{Example}[section]
\newtheorem{definition}[problem]{Definition}
\newcommand{\BEQA}{\begin{eqnarray}}
\newcommand{\EEQA}{\end{eqnarray}}

\theoremstyle{remark}

% Marks the beginning of the document
\begin{document}
\bibliographystyle{IEEEtran}
\vspace{3cm}

\title{2011-CE-"40-52"}
\author{ai24btech11028 - Ronit Ranjan}
\maketitle
\bigskip


\begin{enumerate}


    \item A spillway discharges flood flow at a rate of $9 m^3/s$ per metre width. If the depth of flow on the horizontal apron at the toe of the spillway is $46$ cm, the tail water depth needed to form a hydraulic jump is approximately given by which of the following options? \hfill{[2011-CE]}
    \begin{enumerate}
        \item $2.54$m
        \item $4.90$m
        \item $5.77$m
        \item $6.23$m
    \end{enumerate}

    \item In an aquifer extending over $150$ hectare, the water table was $20$m below ground level. Over a period of time the water table dropped to $23$m below the ground level. If the porosity of aquifer is $0.40$ and the specific retention is $0.15$, what is the change in ground water storage of the aquifer? \hfill{[2011-CE]}
    \begin{enumerate}
        \item $67.5$ ha-m
        \item $112.5$ ha-m
        \item $180.0$ ha-m
        \item $450.0$ ha-m
    \end{enumerate}

    \item Total suspended particulate matter (TSP) concentration in ambient air is to be measured using a high volume sampler. The filter used for this purpose had an initial dry weight of $9.787$ g. The filter was mounted in the sampler and the initial air flow rate through the filter was set at $1.5 m^3$/min. Sampling continued for 24 hours. The airflow after 24 hours was measured to be $1.4 m^3$/min. The dry weight of the filter paper after 24-hour sampling was $10.283$ g. Assuming a linear decline in the air flow rate during sampling, what is the 24-hour average TSP concentration in the ambient air? \hfill{[2011-CE]}
    \begin{enumerate}
        \item 59.2 $\mu$g/m$^3$
        \item 118.6 $\mu$g/m$^3$
        \item 237.5 $\mu$g/m$^3$
        \item 574.4 $\mu$g/m$^3$
    \end{enumerate}

    \item Chlorine gas (8 mg/L as Cl$_2$) was added to a drinking water sample. If the free chlorine residual and pH was measured to be 2 mg/L (as Cl$_2$) and 7.5, respectively, what is the concentration of residual OCl$^-$ ions in the water? Assume that the chlorine gas added to the water is completely converted to HOCl and OCl$^-$. Atomic Weight of Cl: 35.5 
    Given: OCl$^-$ + H$^+$ $\rightleftharpoons$ HOCl,   $K = 10^{7.5}$ \hfill{[2011-CE]}
    \begin{enumerate}
        \item 1.408 $\times$ 10$^{-5}$ moles/L
        \item 2.817 $\times$ 10$^{-5}$ moles/L
        \item 5.634 $\times$ 10$^{-5}$ moles/L
        \item 1.127 $\times$ 10$^{-5}$ moles/L
    \end{enumerate}

    \item If the jam density is given as $k_j$ and the free flow speed is given as $u_f$, the maximum flow for a linear traffic speed-density model is given by which of the following options? \hfill{[2011-CE]}
    \begin{enumerate}
        \item $\frac{k_j}{4} \times u_f$
        \item $\frac{k_j}{3} \times u_f$
        \item $\frac{2k_j}{5} \times u_f$
        \item $\frac{2k_j}{3} \times u_f$
    \end{enumerate}

    \item If $v$ is the initial speed of a vehicle, $g$ is the gravitational acceleration, $G$ is the upward longitudinal slope of the road, and $f_r$ is the coefficient of rolling friction during braking, the braking distance (measured horizontally) for the vehicle to stop is \hfill{[2011-CE]}

    \begin{enumerate}
        \item $\frac{v^2}{g(G + f_r)}$
        \item $\frac{v^2}{2g(G + f_r)}$
        \item $\frac{vg}{(G + f_r)}$
        \item $\frac{vf_r}{(G + g)}$
    \end{enumerate}

    \item The cumulative arrival and departure curve of one cycle of an approach lane of a signalized intersection is shown in the adjoining figure. The cycle time is $50$s and the effective red time is $30$s and the effective green time is$20$s. What is the average delay? \hfill{[2011-CE]}
    \begin{figure}[!ht]
    \centering
    \resizebox{0.5\textwidth}{!}{%
    \begin{circuitikz}
    \tikzstyle{every node}=[font=\small]
    \draw  (1.5,13.75) rectangle (10.25,5.75);
    \draw [dashed] (3,13.75) -- (3,5.75);
    \draw [dashed] (4.75,13.75) -- (4.75,5.75);
    \draw [dashed] (7,13.75) -- (7,5.75);
    \draw [dashed] (8.75,13.75) -- (8.75,5.75);
    \draw [dashed] (1.5,12) -- (10.25,12);
    \draw [dashed] (1.5,9.75) -- (10.25,9.75);
    \draw [dashed] (1.5,7.75) -- (10.25,7.75);
    \node [font=\tiny, rotate around={7:(0,0)}] at (1.25,13.75) {40};
    \node [font=\scriptsize, rotate around={7:(0,0)}] at (1.25,12) {30};
    \node [font=\scriptsize, rotate around={7:(0,0)}] at (1.25,9.75) {20};
    \node [font=\scriptsize, rotate around={7:(0,0)}] at (1.25,7.75) {10};
    \node [font=\scriptsize, rotate around={7:(0,0)}] at (3,5.5) {10};
    \node [font=\scriptsize, rotate around={7:(0,0)}] at (4.75,5.5) {20};
    \node [font=\scriptsize, rotate around={7:(0,0)}] at (7,5.5) {30};
    \node [font=\scriptsize, rotate around={7:(0,0)}] at (8.75,5.5) {40};
    \node [font=\scriptsize, rotate around={7:(0,0)}] at (10.25,5.5) {50};
    \draw [short] (1.5,5.75) -- (10.25,13.75);
    \draw [short] (7,5.75) -- (10.25,13.75);
    \node[draw, rectangle] [font=\large] at (5.7,11.75) {Cummulative arrival};
    \node[draw, rectangle] [font=\large] at (10.25,7.0) {Cummualative departure};
    \node [font=\small] at (6,5) {\textbf{Time(s)}};
    \node [font=\small, rotate around={90:(0,0)}] at (0.5,10) {\textbf{Cummulative arrival or departure}};
    \node [font=\small, rotate around={90:(0,0)}] at (0.75,10) {\textbf{(No. of Vehicles)}};
    \end{circuitikz}
    }%

\label{fig:my_label}
\end{figure}
   \begin{enumerate}
       \item $15$s
       \item $25$s
       \item $35$s
       \item $45$s
       
   \end{enumerate}

\item The observations from a closed loop transverse around an obstacle are \hfill{[2011-CE]}


\begin{table}[H]
\centering 
\begin{tabular}{|l|l|l|l|}  
\hline
Segment & \begin{tabular}[c]{@{}l@{}}Observation \\ from station\end{tabular} & Length (m) & \begin{tabular}[c]{@{}l@{}}Azimuth (clockwise\\ from magnetic north)\end{tabular} \\ \hline
PQ      & P                                                                   & Missing    & 33.7500                                                                           \\ \hline
QR      & Q                                                                   & 300.000    & 86.3847                                                                           \\ \hline
RS      & R                                                                   & 354.524    & 169.3819                                                                          \\ \hline
ST      & S                                                                   & 450.000    & 243.9003                                                                          \\ \hline
TP      & T                                                                   & 268.000    & 317.5000                                                                          \\ \hline
\end{tabular}
\end{table}
What is the value of the missing measurement(rounded off to the nearest 10mm)?
    \begin{enumerate}
        \item $396.86$ m
        \item $396.79$ m
        \item $396.08$ m
        \item $396.94$ m
    \end{enumerate}
\section*{Common Data Questions}
    \subsection*{Common Data for Questions 48 and 49:}
    A sand layer found at sea floor under 20 m water depth is characterized with relative density = 40\%, maximum void ratio = 1.0, minimum void ratio = 0.5, and specific gravity of soil solids = 2.67. Assume the specific gravity of sea water to be 1.03 and the unit weight of fresh water to be 9.81 kN/m$^3$.
    \item What would be the effective stress (rounded off to the nearest integer value of kPa) at 30 m depth into the sand layer? \hfill{[2011-CE]}
    \begin{enumerate}
        \item $77$ kPa
        \item $273$ kPa
        \item $268$ kPa
        \item $281$ kPa
    \end{enumerate}

    \item What would be the change in the effective stress (rounded off to the nearest integer value of kPa) at 30 m depth into the sand layer if the sea water level permanently rises by 2 m? \hfill{[2011-CE]}
    \begin{enumerate}
        \item $19$ kPa
        \item $0$ kPa
        \item $21$ kPa
        \item $22$ kPa
    \end{enumerate}

    \subsection*{Common Data for Questions 50 and 51:}
    The ordinates of a 2-h unit hydrograph at 1 hour intervals starting from time $t = 0$, are 0, 3, 8, 6, 3, 2 and 0 m$^3$/s. Use trapezoidal rule for numerical integration, if required.

    \item What is the catchment area represented by the unit hydrograph? \hfill{[2011-CE]}
    \begin{enumerate}
    \item 1.00 km$^2$
    \item 2.00 km$^2$
    \item 7.92 km$^2$
    \item 8.64 km$^2$
    \end{enumerate}

    \item A storm of 6.6 cm occurs uniformly over the catchment in 3 hours. If $\phi$-index is equal to 2 mm/h and base flow is 5 m$^3$/s, what is the peak flow due to the storm? \hfill{[2011-CE]}
    \begin{enumerate}
    \item 41.0 m$^3$/s
    \item 43.4 m$^3$/s
    \item 53.0 m$^3$/s
    \item 56.2 m$^3$/s
    \end{enumerate}


\section*{Linked Answer Questions}
    \subsection*{Statement for Linked Answer Questions 52 and 53: }
    A rigid beam is hinged at one end and supported on linear elastic springs (both having a stifness of 'k') at points '1' and '2', and an inclined load acts at '2', as shown. \hfill{[2011-CE]}
    \begin{figure}[!ht]
\centering
\resizebox{0.5\textwidth}{!}{%
\begin{circuitikz}
\tikzstyle{every node}=[font=\large]
\draw [short] (1.25,10.75) -- (1.25,5.75);
\draw [short] (1.25,10.75) -- (2,10.75);
\draw [short] (2,10.75) -- (2.5,9.5);
\draw [short] (2.5,9.5) -- (5,7);
\draw [short] (5,7) -- (5,5.75);
\draw [short] (5,5.75) -- (1.25,5.75);
\draw [short] (2.75,9.25) -- (3.75,9.25);
\draw [short] (3.75,8.25) -- (3.75,9.25);
\draw [short] (3.75,9.25) -- (3.75,9.5);
\draw [short] (3.75,9.5) -- (15,9.5);
\draw [short] (3.75,9.25) -- (15,9.25);
\draw [short] (15,9.5) -- (15,9.25);
\draw [short] (8.75,8.25) -- (10,8.25);
\draw [short] (13.75,8.25) -- (15,8.25);
\draw (9.25,9.25) to[R] (9.25,8.25);
\draw (14.25,9.25) to[R] (14.25,8.25);
\draw [->, >=Stealth] (13.25,11.25) -- (15,9.5);
\draw [<->, >=Stealth] (3.75,5.25) -- (9.5,5.25);
\draw [<->, >=Stealth] (9.5,5.25) -- (15,5.25);
\draw [->, >=Stealth] (2.25,5.25) -- (3.25,8.5);
\draw [->, >=Stealth] (2.5,12.25) -- (3.75,9.5)node[pos=0.5, fill=white]{Hinge};
\node [font=\large] at (1.75,5) {Fixed};
\node [font=\large] at (6.5,5.75) {l};
\node [font=\large] at (12,5.75) {l};
\draw [->, >=Stealth] (11.25,7.25) -- (9.25,8.25);
\draw [->, >=Stealth] (13,7.5) -- (14.5,8.25);
\node [font=\large] at (12,7.5) {Fixed};
\node [font=\large] at (13.75,10) {$45\circ$};
\draw  (9,10) rectangle (9.5,9.5);
\draw  (15,10.25) rectangle (15.5,9.75);
\node [font=\large] at (9.25,9.75) {\textbf{1}};
\node [font=\large] at (15.25,10) {\textbf{2}};
\end{circuitikz}
}%

\label{fig:my_label}
\end{figure}

\item Which of the following options represent the deflections $\delta_{1}$ and $\delta_{2}$ at the points 1 and 2 \hfill{[2011-CE]}

\begin{enumerate}
    \item $\delta_1 = \frac{2}{5} \frac{2P}{k}$ and $\delta_2 = \frac{4}{5} \frac{2P}{k}$
    \item $\delta_1 = \frac{2}{5} \frac{P}{k}$ and $\delta_2 = \frac{4}{5} \frac{P}{k}$
    \item $\delta_1 = \frac{2}{5} \frac{P}{\sqrt{2}k}$ and $\delta_2 = \frac{4}{5} \frac{P}{\sqrt{2}k}$
    \item $\delta_1 = \frac{2}{5} \frac{\sqrt{2}P}{k}$ and $\delta_2 = \frac{4}{5} \frac{\sqrt{2}P}{k}$
\end{enumerate}
    




    
\end{enumerate}


\end{document}
