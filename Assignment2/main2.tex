%iffalse
\let\negmedspace\undefined
\let\negthickspace\undefined
\documentclass[journal,12pt,twocolumn]{IEEEtran}
\usepackage{cite}
\usepackage{amsmath,amssymb,amsfonts,amsthm}
\usepackage{algorithmic}
\usepackage{graphicx}
\usepackage{textcomp}
\usepackage{xcolor}
\usepackage{txfonts}
\usepackage{listings}
\usepackage{enumitem}
\usepackage{mathtools}
\usepackage{gensymb}
\usepackage{comment}
\usepackage[breaklinks=true]{hyperref}
\usepackage{tkz-euclide} 
\usepackage{listings}
\usepackage{gvv}                                        
%\def\inputGnumericTable{}                                 
\usepackage[latin1]{inputenc}                                
\usepackage{color}                                            
\usepackage{array}                                            
\usepackage{longtable}                                       
\usepackage{calc}                                             
\usepackage{multirow}                                         
\usepackage{hhline}                                           
\usepackage{ifthen}                                           
\usepackage{lscape}
\usepackage{tabularx}
\usepackage{array}
\usepackage{float}
\usepackage{multicol}


\newtheorem{theorem}{Theorem}[section]
\newtheorem{problem}{Problem}
\newtheorem{proposition}{Proposition}[section]
\newtheorem{lemma}{Lemma}[section]
\newtheorem{corollary}[theorem]{Corollary}
\newtheorem{example}{Example}[section]
\newtheorem{definition}[problem]{Definition}
\newcommand{\BEQA}{\begin{eqnarray}}
\newcommand{\EEQA}{\end{eqnarray}}
\newcommand{\define}{\stackrel{\triangle}{=}}
\theoremstyle{remark}
\newtheorem{rem}{Remark}

% Marks the beginning of the document
\begin{document}
\bibliographystyle{IEEEtran}
\vspace{3cm}

\title{JEE Chapter 19 CD}
\author{AI24BTECH11028 - Ronit Ranjan}
\maketitle
\newpage
\bigskip
\section*{C. MCQ with one correct answer}
\begin{enumerate}
\item A solution of the differential equation \hfill \brak{1999 - 2 Marks}
\begin{align}
\brak{\frac{dy}{dx}}^2 - x \frac{dy}{dx} + y = 0 
\end{align}
\begin{multicols}{2}
\begin{enumerate}
    \item $y = 2$
    \item $y = 2x -4$
    \item $y = 2x$
    \item $y = 2x^{2}-4$
\end{enumerate}
\end{multicols}

\item If $x^2 + y^2 = 1$, then \hfill \brak{2000S}
\begin{multicols}{2}
\begin{enumerate}
    \item  $yy"-2\brak{y'^{2}}+1 = 0$
    \item  $yy"+\brak{y'^{2}}+1 = 0$
    \item  $yy"-\brak{y'^{2}}-1 = 0$
    \item  $yy"+2\brak{y'^{2}}+1 = 0$
\end{enumerate}
\end{multicols}

\item If $y\brak{t}$ is a solution of $\brak{1 + t} \frac{dy}{dt} - ty = 1$ and $y\brak{0} = -1$, then $y(1)$ is equal to\hfill \brak{2003S}
\begin{multicols}{2}
\begin{enumerate}
    \item $-\frac{1}{2}$
    \item $e - \frac{1}{2}$
    \item $e + \frac{1}{2}$
    \item $\frac{1}{2}$
\end{enumerate}
\end{multicols}

\item If $y = y\brak{x}$ and $\frac{2 + \sin x}{y + 1}\brak{\frac{dy}{dx}} = -\cos x$, $y\brak{0} = 1$, then y\brak{\frac{\pi}{2}} \hfill \brak{2004S}

\begin{multicols}{2}
\begin{enumerate}
    \item $\frac{1}{3}$
    \item $-\frac{1}{3}$
    \item $\frac{2}{3}$
    \item $1$
\end{enumerate}
\end{multicols}

\item If $y = y(x)$ and it follows the relation $x \cos y + y \cos x = \pi$ then $y''(0) =$ \hfill \brak{2005S}

\begin{multicols}{2}
\begin{enumerate}
    \item $1$
    \item $-1$
    \item $\pi - 1$
    \item $-\pi$
\end{enumerate}  
\end{multicols}

\item The solution of primitive integral equation $\brak{x^2 + y^2}dy = xy \, dx$ is $y = y(x)$. If $y(1) = 1$ and $x_0 = e$, then $x_0$ is equal to \hfill \brak{2005S}

\begin{multicols}{2}
\begin{enumerate}
    \item $\sqrt{2\brak{e^2 - 1}}$
    \item $\sqrt{3e}$ 
    \item $\sqrt{2\brak{e^2 + 1}}$
    \item $\sqrt{\frac{e^2 + 1}{2}}$
\end{enumerate}
\end{multicols}

\item For the primitive integral equation $y dx + y^2 dy = x \, dy$; $x \in \mathbb{R}, y > 0, y = y(x), y(1) = 1$, then $y(-3)$ is \hfill \brak{2005S}

\begin{multicols}{2}
\begin{enumerate}
    \item $3$
    \item $1$ 
    \item $2$
    \item $5$
\end{enumerate}
\end{multicols}

\item The differential equation $\frac{dy}{dx} = \frac{\sqrt{1-y^2}}{y}$ determines a family of circles with \hfill \brak{2005S}
\begin{enumerate}[label=(\alph*)]
    \item variable radii and a fixed centre at \brak{0,1}
    \item variable radii and a fixed centre at \brak{0, -1}
    \item fixed radius 1 and variable centres along the x-axis
    \item fixed radius 1 and variable centres along the y-axis
\end{enumerate}
\item The function $y = f(x)$ is the solution of the differential equation \hfill \brak{JEE Adv.  2014}
\begin{align}
\frac{dy}{dx} + \frac{xy}{x^2-1} = \frac{x^4 + 2x}{\sqrt{1-x^2}}
\end{align}
in $(-1, 1)$ satisfying $f(0) = 0$. Then
\begin{align}
\int_{-\frac{\sqrt{3}}{2}}^{\frac{\sqrt{3}}{2}} f(x) \, dx is
\end{align}

\begin{multicols}{2}
\begin{enumerate}
    \item $\frac{\pi}{3} - \frac{\sqrt{3}}{2}$
    \item $\frac{\pi}{6} - \frac{\sqrt{3}}{4}$
    \item $\frac{\pi}{3} - \frac{\sqrt{3}}{4}$ 
    \item $\frac{\pi}{6} - \frac{\sqrt{3}}{2}$
\end{enumerate}
\end{multicols}

\item If $y = y(x)$ satisfies the differential equation\hfill \brak{JEE Adv. 2018}
\begin{align}
8\sqrt{x} \brak{\sqrt{9+\sqrt{x}}} dy = \brak{\sqrt{4+\sqrt{9+\sqrt{x}}}}^{-1} dx, \, x>0
\end{align}

and y(0) = $\sqrt{7}$, \text{ then } y(256) = 

\begin{multicols}{2}
\begin{enumerate}
    \item $3$
    \item $16$ 
    \item $9$
    \item $80$
\end{enumerate}
\end{multicols}
\end{enumerate}
\section*{D. MCQ with one or more than correct answer}
\begin{enumerate}
    
\item The order of the differential equation whose general solution is given by
$y = \brak{C_1 + C_2} \cos \brak{x + C_3} - C_4 e^{x+C_5}$, where $C_1, C_2, C_3, C_4, C_5$ are arbitrary constants, is \hfill \brak{1998 - 2 Marks}

\begin{multicols}{2}
\begin{enumerate}
    \item $5$
    \item $3$ 
    \item $4$
    \item $2$ 
\end{enumerate} 
\end{multicols}

\item  The differential equation representing the family of curves
$y^2 = 2c \brak{x + \sqrt{c}}$, where c  is a positive parameter, is of \hfill \brak{1999 - 3 Marks}

\begin{multicols}{2}
\begin{enumerate}
    \item order $1$ 
    \item degree $3$
    \item order $2$
    \item degree $4$
\end{enumerate}
\end{multicols}

\item  A curve $ y = f(x) $ passes through $ \brak{1,1} $ and at $ P\brak{x, y} $, the tangent cuts the $ x $-axis and $ y $-axis at $ A $ and $ B $ respectively such that $ BP : AP = 3 : 1 $, then \hfill \brak{2006- 5M,-1}
\begin{enumerate}
    \item equation of curve is $ xy' - 3y = 0 $
    \item normal at $ \brak{1, 1} $ is $ x + 3y = 4 $
    \item curve passes through $ \brak{2, \frac{1}{8}} $
    \item equation of curve is $xy' + 3y = 0 $
\end{enumerate}

\item  If $y\brak{x}$ satisfies the differential equation $y' - y \tan x = 2x \sec x $ and $y\brak{0} = 0 $, then \hfill \brak{2012}

\begin{multicols}{2}
\begin{enumerate}
    \item $ y\brak{\frac{\pi}{4}} = \frac{\pi^2}{8\sqrt{2}} $
    \item $ y\brak{\frac{\pi}{3}} = \frac{\pi^2}{9} $
    \item $ y\brak{\frac{\pi}{4}} = \frac{\pi^2}{18} $
    \item $ y\brak{\frac{\pi}{3}} = \frac{4\pi}{3} + \frac{2\pi^2}{3\sqrt{3}} $
\end{enumerate}
\end{multicols}

\item A curve passes through the point $ \brak{1,\frac{\pi}{6}} $. Let the slope of the curve at each point $ \brak{x, y} $ be $ \frac{y}{x} + \sec\brak{\frac{y}{x}}, x > 0 $. Then the equation of the curve is \hfill \brak{JEE Adv. 2013}

\begin{multicols}{2}
\begin{enumerate}
    \item $ \sin\brak{\frac{y}{x}} = \log x + \frac{1}{2} $
    \item $\sec\brak{\frac{2y}{x}} = \log x + 2$
    \item $\cos\sec\brak{\frac{y}{x}} = \log x + 2 $
    \item $ \cos\brak{\frac{2y}{x}} = \log x + \frac{1}{2} $
\end{enumerate}
\end{multicols}
\end{enumerate}

\end{document}
