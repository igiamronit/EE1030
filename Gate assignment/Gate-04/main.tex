%iffalse
\let\negmedspace\undefined
\let\negthickspace\undefined
\documentclass[journal,12pt,onecolumn]{IEEEtran}
\usepackage{cite}
\usepackage{amsmath,amssymb,amsfonts,amsthm}
\usepackage{algorithmic}
\usepackage{graphicx}
\usepackage{textcomp}
\usepackage{xcolor}
\usepackage{txfonts}
\usepackage{listings}
\usepackage{enumitem}
\usepackage{mathtools}
\usepackage{gensymb}
\usepackage{comment}
\usepackage[breaklinks=true]{hyperref}
\usepackage{tkz-euclide} 
\usepackage{listings}
\usepackage{gvv}                                        
%\def\inputGnumericTable{}                                 
\usepackage[latin1]{inputenc}                                
\usepackage{color}                                            
\usepackage{array}                                            
\usepackage{longtable}                                       
\usepackage{calc}                                             
\usepackage{multirow}                                         
\usepackage{hhline} 
\usepackage{ifthen}                                           
\usepackage{lscape}
\usepackage{tabularx}
\usepackage{array}
\usepackage{float}
\usepackage{multicol}
\usepackage{tikz}
\usetikzlibrary{graphs}
\usepackage{circuitikz}



\newtheorem{theorem}{Theorem}[section]
\newtheorem{problem}{Problem}
\newtheorem{proposition}{Proposition}[section]
\newtheorem{lemma}{Lemma}[section]
\newtheorem{corollary}[theorem]{Corollary}
\newtheorem{example}{Example}[section]
\newtheorem{definition}[problem]{Definition}
\newcommand{\BEQA}{\begin{eqnarray}}
\newcommand{\EEQA}{\end{eqnarray}}

\theoremstyle{remark}

% Marks the beginning of the document
\begin{document}
\bibliographystyle{IEEEtran}
\vspace{3cm}

\title{2019-ME-"1-13"}
\author{ai24btech11028 - Ronit Ranjan}
\maketitle
\bigskip

\begin{enumerate}
    \item John Thomas, an \underline{\hspace{2cm}} writer, passed away in 2018
    \begin{enumerate}
        \item imminent
        \item porminent
        \item eminent
        \item dominant
    \end{enumerate}

    \item \underline{\hspace{2cm}} I permitted him to leave, I wouldn't have had any problem with him being absent,\underline{\hspace{2cm}} I?
    \begin{enumerate}
        \item Had, wouldn't
        \item Have, would
        \item Had, would
        \item Have, wouldn't
    \end{enumerate}

    \item A worker noticed that the hour hand on the factory clock had moved by 225 degrees during her stay at the factory. For how long did she stay in the factory?
    \begin{enumerate}
        \item 3.75 hours
        \item 4 hours and 15 mins
        \item 8.5 hours
        \item 7.5 hours
    \end{enumerate}

    \item The sum and product of two integers are 26 and 165 respectively. The difference between these two integers is\underline{\hspace{2cm}}.
    \begin{enumerate}
        \item 2
        \item 3
        \item 4
        \item 6
    \end{enumerate}

    \item The minister avoided any mention of the issue of women's reservation in the private sector. He was accused of \underline{\hspace{2cm}}the issue.
    \begin{enumerate}
        \item collaring
        \item skirting
        \item tying
        \item belting
    \end{enumerate}

    \subsection*{Q.\ref{q6} - Q.\ref{q10} carry two marks each}

    \item \label{q6} Under a certain legal system, prisoners are allowed to make one statement. If their statement turns out to be true then they are hanged. If the statement turns out to be false then they are shot. One prisoner made a statement and the judge had no option but to set him free. Which one of the following could be that statement?
    \begin{enumerate}
        \item I did not commit the crime
        \item I committed the crime
        \item I will be shot
        \item You committed the crime
    \end{enumerate}

    \item \label{Q7} A person divided an amount of Rs. 100,000 into two parts and invested in two different schemes. In one he got 10\% profit and in the other he got 12\%. If the profit percentages are interchanged with these investments he would have got Rs. 120 less. Find the ratio between his investments in the two schemes.
    \begin{enumerate}
        \item 9 : 16
        \item 11 : 14
        \item 37 : 63
        \item 47 : 53
    \end{enumerate}

    \item \label{Q8} Congo was named by Europeans. Congo's dictator Mobuto later changed the name of the country and the river to Zaire with the objective of Africanising names of the persons and spaces. However, the name Zaire was a Portuguese alteration of \textit{Nzadi o Nzere}, a local African term meaning 'River that swallows Rivers'. Zaire was the Portuguese name for the Congo river in the 16th and 17th centuries.
    Which of the following statements can be inferred from the paragraph above?
    \begin{enumerate}
        \item Mobuto was not entriely successful in Africanising the name of his country
        \item The term \textit{Nzadi o Nzere} was of Portuguese origin
        \item Mobuto's desire to Africanise names was prevented by the Portuguese.
        \item As a dictator Mobuto ordered the Portuguese to alter the name of the river to Zaire
    \end{enumerate}

    \item \label{Q9}A firm hires the employees at five different skill levels P, Q, R, S, T. The shares of employement at these skill levels of total employment in 2010 is given in the pie chart as shown. There were a total of 600 employment in 2010 and the total employment increased by 15\% from 2010 to 2016. The total employment at skill levels P, Q and R remained unchanged during this period. If the employment at skill level S increased by 40\% from 2010 to 2016, how many employees were at skill level T in 2016?
    \begin{figure}[!ht]
\centering
\resizebox{0.5\textwidth}{!}{%
\begin{circuitikz}
\tikzstyle{every node}=[font=\normalsize]
\draw  (4.5,9.5) circle (4cm);
\draw [short] (4.5,9.25) -- (4.5,13.5);
\draw [short] (4.5,9.25) -- (7.75,11.75);
\draw [short] (4.5,9.25) -- (6.25,6);
\draw [short] (4.5,9.25) -- (1,7.5);
\draw [short] (4.5,9.25) -- (3.25,13.25);
\node [font=\large] at (6.75,8.5) {Q=25};
\node [font=\large] at (3.75,7) {R=25};
\node [font=\large] at (2.25,10) {S=25};
\node [font=\large] at (4,12.25) {T=5};
\node [font=\large] at (5.5,11.75) {P=20};
\node [font=\normalsize] at (4.25,4.75) {Percentage share of skills in 2010};
\end{circuitikz}
}%

\label{fig:my_label}
\end{figure}
\begin{enumerate}
    \item 30
    \item 35
    \item 60
    \item 72
\end{enumerate}
    \item \label{q10}M and N had four children P, Q, R and S. Of them, only P and R were married. They had children X and Y respectively. If Y is a legitimate child of W, which one the following statements is necessarily is FALSE?
    \begin{enumerate}
        \item M is the grandmother of Y
        \item R is the father of Y
        \item W is the wife of R
        \item W is the wife of P
    \end{enumerate}
\end{enumerate}


    \subsection*{Q.\ref{Q1} - Q.\ref{Q25} carry two marks each} 
    
    \begin{enumerate}
        \item \label{Q1} Consider the matrix
        \begin{align*}
        P = 
            \begin{bmatrix}
            1 & 1 & 0 \\
            0 & 1 & 1 \\
            0 & 0 & 1
            \end{bmatrix}
        \end{align*}
        The number of distinct eigenvalues of P is
        \begin{enumerate}
            \item 0
            \item 1
            \item 2
            \item 3
        \end{enumerate}

        \item A parabola $x = y^2$ with $0 \leq x \leq 1$ is shown in the figure. The volume of the solid of rotation obtained by rotating the shaded area by 360\degree around the x-axis is
        \begin{figure}[!ht]
\centering
\resizebox{0.3\textwidth}{!}{%
\begin{circuitikz}
\tikzstyle{every node}=[font=\large]
\draw [line width=0.6pt, short] (1.25,7.75) -- (1.5,8.25);
\draw [line width=0.6pt, short] (1.5,8.25) -- (1.75,8.5);
\draw [line width=0.6pt, short] (1.75,8.5) -- (2,8.75);
\draw [line width=0.6pt, short] (2,8.75) -- (2.5,9);
\draw [line width=0.6pt, short] (2.5,9) -- (3.25,9.25);
\draw [line width=0.6pt, short] (3.25,9.25) -- (4,9.5);
\draw [line width=0.6pt, short] (4,9.5) -- (4.75,9.75);
\draw [line width=0.6pt, short] (4.75,9.75) -- (5.75,9.75);
\draw [->, >=Stealth] (1.25,7.75) -- (7,7.75);
\draw [->, >=Stealth] (1.25,7.75) -- (1.25,11.25);
\draw [line width=0.2pt, short] (5.75,9.75) -- (5.75,7.75);
\node [font=\large] at (1,11.5) {y};
\node [font=\large] at (7,7.5) {x};
\node [font=\large] at (3,9.75) {$x = y^2$};
\node [font=\large] at (1.25,7.25) {$0$};
\node [font=\large] at (5.75,7.25) {$1$};
\end{circuitikz}
}%

\label{fig:my_label}
\end{figure}
\begin{enumerate}
    \item $\frac{\pi}{4}$
    \item $\frac{\pi}{2}$
    \item $\pi$
    \item $2\pi$
\end{enumerate}

\item \label{Q3} For the equation $frac{dy}{dx} + 7x^2 y = 0$ if $y\brak{0} = \frac{3}{7}$, then the value of $y\brak{1}$ is
\begin{enumerate}
    \item $\frac{7}{3} e^{-7/3}$
    \item $\frac{7}{3} e^{-3/7}$
    \item $\frac{3}{7} e^{-7/3}$
    \item $\frac{3}{7} e^{-3/7}$
\end{enumerate}
    \end{enumerate}

\end{document}

