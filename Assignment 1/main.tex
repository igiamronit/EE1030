%iffalse
\let\negmedspace\undefined
\let\negthickspace\undefined
\documentclass[journal,12pt,twocolumn]{IEEEtran}
\usepackage{cite}
\usepackage{amsmath,amssymb,amsfonts,amsthm}
\usepackage{algorithmic}
\usepackage{graphicx}
\usepackage{textcomp}
\usepackage{xcolor}
\usepackage{txfonts}
\usepackage{listings}
\usepackage{enumitem}
\usepackage{mathtools}
\usepackage{gensymb}
\usepackage{comment}
\usepackage[breaklinks=true]{hyperref}
\usepackage{tkz-euclide} 
\usepackage{listings}
\usepackage{gvv}                                        
%\def\inputGnumericTable{}                                 
\usepackage[latin1]{inputenc}                                
\usepackage{color}                                            
\usepackage{array}                                            
\usepackage{longtable}                                       
\usepackage{calc}                                             
\usepackage{multirow}                                         
\usepackage{hhline}                                           
\usepackage{ifthen}                                           
\usepackage{lscape}
\usepackage{tabularx}
\usepackage{array}
\usepackage{float}


\newtheorem{theorem}{Theorem}[section]
\newtheorem{problem}{Problem}
\newtheorem{proposition}{Proposition}[section]
\newtheorem{lemma}{Lemma}[section]
\newtheorem{corollary}[theorem]{Corollary}
\newtheorem{example}{Example}[section]
\newtheorem{definition}[problem]{Definition}
\newcommand{\BEQA}{\begin{eqnarray}}
\newcommand{\EEQA}{\end{eqnarray}}
\newcommand{\define}{\stackrel{\triangle}{=}}
\theoremstyle{remark}
\newtheorem{rem}{Remark}

% Marks the beginning of the document
\begin{document}
\bibliographystyle{IEEEtran}
\vspace{3cm}

\title{JEE Chapter 3 A,B}
\author{ai24btech11028 - Ronit Ranjan}
\maketitle
\newpage
\bigskip
\section*{A. Fill in the blanks}
\begin{enumerate}
    \item The coefficient of $x^{99}$ in the polynomial\\ (x-1)(x-2)....(x-100) is.......\hfill (1982-2 Marks)
      
    \item If $2+i\sqrt{3}$ is a root of the equation $x^2 + px +q =0$,where p and q are real, then(p,q)=(........ , ........).\hfill (1982 - 2 Marks)
    
    \item If the product of the roots of the equation\\ $x^2 -3kx +2e^{2\ln{k}} -1=0$ is $7$, then the roots are real for k =........\hfill (1984 - 2 Marks)
    
    \item If the quadratic equation $x^2 + ax +b=0$ and $x^2 + bx + c=0 (a \ne b)$have a common root then value of a+b is......\hfill (1986 - 2 Marks)
    
    \item The solution of equation $\log_{7}\log_{5}(\sqrt{x+5}+\sqrt{x} = 0$ is......\hfill (1986 - 2 Marks)
    
    \item If $x<0, y,0, x + y + \frac{x}{y} = \frac{1}{2}$ and $(x+y)(\frac{x
    }{y}) = -\frac{1}{2}$, then x= ....... and y= .......\hfill (1990 - 2 Marks)
    
    \item Let n and k be such positive numbers such that $n \geq \frac{(k)(k+1)}{2}$ . The number of solutions $(x_1,x_2,....x_k), x_1 \geq 1, x_2 \geq 2,...,x_k \geq k, $ all integers, satisfying $x_1+x_2+...x_k = n$, is........  \hfill (1996 - 2Marks) 
    \item The sum of all the real roots of the equation\\$|x-2|^2+|x-2|-2 = 0$ is \hfill (1997 - 2 Marks)
\end{enumerate}
\section*{B. True / False}
\begin{enumerate}
    \item For every integer $n>1$, the inequality $(n!)^\frac{1}{n} < \frac{n+1}{2}$ holds. \hfill (1981 - 2 Marks)
    \item The equation $2x^2 + 3x + 1 = 0$ has an irrational root. \hfill (1983 - 1 Mark)
    \item If $a<b<c<d$, then the roots of the equation $(x-a)(x-c)+2(x-b)(x-d)=0$ are real and distinct. \hfill (1984 - 1 Mark)
    \item If $n_1, n_2, ....n_p$ are p positive integers, whose sum is an even number, then the number of odd integers among them is odd.\hfill (1985 - 1 Mark)
    \item If $P(x) = ax^2+bx+c$ and $Q(x)= -ax^2+dx+c$, where ac $\neq$ 0, then P(x)Q(x)=0 has at least two real roots. \hfill(1985 - 1 Marks)
    \item If x and y are positive real numbers and m,n are any positive integers, then $\frac{x^ny^m}{(1+ x^{2n})(1+ y^{2m})} > \frac{1}{4}$ \hfill (1989 - 1 Mark)
\end{enumerate}
\renewcommand{\thefigure}{\theenumi}
\renewcommand{\thetable}{\theenumi}


\end{document}
