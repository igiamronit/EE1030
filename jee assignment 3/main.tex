%iffalse
\let\negmedspace\undefined
\let\negthickspace\undefined
\documentclass[journal,12pt,twocolumn]{IEEEtran}
\usepackage{cite}
\usepackage{amsmath,amssymb,amsfonts,amsthm}
\usepackage{algorithmic}
\usepackage{graphicx}
\usepackage{textcomp}
\usepackage{xcolor}
\usepackage{txfonts}
\usepackage{listings}
\usepackage{enumitem}
\usepackage{mathtools}
\usepackage{gensymb}
\usepackage{comment}
\usepackage[breaklinks=true]{hyperref}
\usepackage{tkz-euclide} 
\usepackage{listings}
\usepackage{gvv}                                        
%\def\inputGnumericTable{}                                 
\usepackage[latin1]{inputenc}                                
\usepackage{color}                                            
\usepackage{array}                                            
\usepackage{longtable}                                       
\usepackage{calc}                                             
\usepackage{multirow}                                         
\usepackage{hhline} 
\usepackage{ifthen}                                           
\usepackage{lscape}
\usepackage{tabularx}
\usepackage{array}
\usepackage{float}
\usepackage{multicol}


\newtheorem{theorem}{Theorem}[section]
\newtheorem{problem}{Problem}
\newtheorem{proposition}{Proposition}[section]
\newtheorem{lemma}{Lemma}[section]
\newtheorem{corollary}[theorem]{Corollary}
\newtheorem{example}{Example}[section]
\newtheorem{definition}[problem]{Definition}
\newcommand{\BEQA}{\begin{eqnarray}}
\newcommand{\EEQA}{\end{eqnarray}}

\theoremstyle{remark}

% Marks the beginning of the document
\begin{document}
\bibliographystyle{IEEEtran}
\vspace{3cm}

\title{JEE Chapter 3 A,B}
\author{ai24btech11028 - Ronit Ranjan}
\maketitle
\newpage
\bigskip

\section*{Section A}

\begin{enumerate}
    \item The mean and variance of the data $4, 5, 6, 6, 7, 8, x, y$ where $x < y$ are $6$ and $\frac{9}{4}$ respectively. Then $x^4 + y^2$ is equal to 
    \begin{multicols}{2}
    \begin{enumerate}
        \item $162$
        \item $320$
        \item $674$
        \item $420$
    \end{enumerate}
    \end{multicols}

    \item If a point $A\brak{x, y}$ lies in the region bounded by the y-axis, straight lines $2y + x = 6$ and $5x - 6y = 30$, then the probablity that $y<1$ is:
    \begin{multicols}{2}
    \begin{enumerate}
        \item $\frac{1}{6}$
        \item $\frac{5}{6}$
        \item $\frac{2}{3}$
        \item $\frac{6}{7}$
    \end{enumerate}
    \end{multicols}    

    \item The value of $\cot \left( \sum_{n=1}^{50} \tan^{-1} \left( \frac{1}{1+n+n^2} \right) \right)$ is  
    \begin{multicols}{2}
    \begin{enumerate}
        \item $\frac{26}{25}$
        \item $\frac{25}{26}$
        \item $\frac{50}{51}$
        \item $\frac{52}{51}$
    \end{enumerate}
    \end{multicols}    

    \item $\alpha = \sin 36\degree$ is a root of which of the following equation
    \begin{multicols}{2}
    \begin{enumerate}
        \item $10x^4 - 10x^2 -5 = 0$
        \item $16x^4 + 20x^2 -5 = 0$
        \item $16x^4 - 20x^2 +5 = 0$
        \item $16x^4 - 10x^2 +5 = 0$
    \end{enumerate}
    \end{multicols}

    \item Which of the following statement is a tautology?
    \begin{multicols}{2}
    \begin{enumerate}
        \item $((\sim q \cap p)\cap q $
        \item $((\sim q) \cap p) \cap (p \cap (\sim p))$
        \item $((\sim q) \cap p) \cup (p \cup (\sim p))$
        \item $((p \hat q) \hat (\sim(p \cap q))$
    \end{enumerate}
    \end{multicols}

    
    \item Let $ S = \{1, 2, 3, 4, 5, 6, 7, 8, 9, 10\} $. Define the function $ f: S \to S $ as follows:
    \[
    f(n) = 
    \begin{cases}
    2n, & \text{if } n = 1, 2, 3, 4, 5, \\
    2\brak{11 - n}, & \text{if } n = 6, 7, 8, 9, 10.
    \end{cases}
    \]
    Let $ g: S \to S $ be a function such that:
    \[
    g(n) =
    \begin{cases}
    \frac{n+1}{2}, & \text{if } n \text{ is odd}, \\
    11 - \frac{n}{2}, & \text{if } n \text{ is even}.
    \end{cases}
    \]
    Find the value of $ g\brak{10} \left( g\brak{1} + g\brak{2} + g\brak{3} + g\brak{4} + g\brak{5} \right) $.

    \item Let $ \alpha, \beta $ be the roots of the equation:
    \[
    x^2 - 4x + 5 = 0,
    \]
    and let $ \alpha \gamma, \beta \gamma $ be the roots of the equation:
    \[
    x^2 - \brak{3\sqrt{2} + 7\sqrt{3}}x + \brak{7 + 3\sqrt{5}} = 0.
    \]
    If $ \beta + \gamma = 3\sqrt{2} $, then find $ \brak{\alpha + 2\beta + \gamma}^2 $.

    \item Let $ A $ be a matrix of order $ 2 \times 2 $, whose entries are from the set $ \{0, 1, 2, 3, 4, 5\} $. If the sum of all the entries of $ A $ is a prime number $ p $, $ 2 \leq p < 8 $, find the number of such matrices $ A $.

    \item If the sum of the coefficients of all the positive powers of $ x $ in the binomial expansion of 
    \[
    \brak{x + \frac{2}{x}}^n
    \]
    is 939, find the sum of all the possible integral values of $ n $.


    \item Let $ \lfloor t \rfloor $ denote the greatest integer $ \leq t $ and $ \{ t \} $ denote the fractional part of $ t $. Then the integral value of $ \alpha $ for which the left-hand limit of the function:
    \[
    f(x) = \lfloor 1 + x \rfloor + \frac{\alpha x^{3/2} \{ x \} - 1}{2 \lfloor x \rfloor + \{ x \}}
    \]
    at $ x = 0 $ is equal to $ \alpha - \frac{4}{3} $.

    \item If $ y(x) = x^x $, $ x > 0 $, then find the value of:
    \[
    \frac{d^2x}{dy^2} \text{ at } x = 1.
    \]

    \item If the area of the region $ \{ \brak{x, y}: x^3 + y^3 \leq 1, x + y \geq 0, y \geq 0 \} $ is $ A $, find the value of $ \frac{256A}{\pi} $.

    \item Let $ v $ be the solution of the differential equation:
    \[
    \brak{1 - x^2} \frac{dy}{dx} = \brak{xy + \brak{x^3 + 2}\sqrt{1 - x^2}}, -1 < x < 1,
    \]
    and $ y\brak{0} = 0 $. If:
    \[
    \int_{-1/2}^{1/2} \sqrt{1 - x^2} y\brak{0} dx = k,
    \]
    then $ k^{-1} $ is equal to:

    \item Let a circle $ C $ of radius 5 lie below the x-axis. The line $ L_1: 4x + 3y - 2 = 0 $ passes through the center $ P $ of the circle $ C $ and intersects the line $ L_2: 3x - 4y - 11 = 0 $ at $ Q $. The line $ L_1 $ touches $ C $ at the point $ Q $. Then the distance of $ P $ from the line $ 5x - 12y + 51 = 0 $ is:
    
    \item Let $ S = \{ E_1, E_2, \dots, E_8 \} $ be a sample space of random experiments such that $ P(E_n) = \frac{n}{36} $ for every $ n = 1, 2, \dots, 8 $. Then the number of elements in the set 
    \[
    \left\{ A \subset S : P(A) \geq \frac{4}{5} \right\}
    \]
    is:
   

    
\end{enumerate}

\end{document}

