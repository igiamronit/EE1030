%iffalse
\let\negmedspace\undefined
\let\negthickspace\undefined
\documentclass[journal,12pt,onecolumn]{IEEEtran}
\usepackage{cite}
\usepackage{amsmath,amssymb,amsfonts,amsthm}
\usepackage{algorithmic}
\usepackage{graphicx}
\usepackage{textcomp}
\usepackage{xcolor}
\usepackage{txfonts}
\usepackage{listings}
\usepackage{enumitem}
\usepackage{mathtools}
\usepackage{gensymb}
\usepackage{comment}
\usepackage[breaklinks=true]{hyperref}
\usepackage{tkz-euclide} 
\usepackage{listings}
\usepackage{gvv}                                        
%\def\inputGnumericTable{}                                 
\usepackage[latin1]{inputenc}                                
\usepackage{color}                                            
\usepackage{array}                                            
\usepackage{longtable}                                       
\usepackage{calc}                                             
\usepackage{multirow}                                         
\usepackage{hhline} 
\usepackage{ifthen}                                           
\usepackage{lscape}
\usepackage{tabularx}
\usepackage{array}
\usepackage{float}
\usepackage{multicol}
\usepackage{tikz}
\usetikzlibrary{graphs}
\usepackage{circuitikz}



\newtheorem{theorem}{Theorem}[section]
\newtheorem{problem}{Problem}
\newtheorem{proposition}{Proposition}[section]
\newtheorem{lemma}{Lemma}[section]
\newtheorem{corollary}[theorem]{Corollary}
\newtheorem{example}{Example}[section]
\newtheorem{definition}[problem]{Definition}
\newcommand{\BEQA}{\begin{eqnarray}}
\newcommand{\EEQA}{\end{eqnarray}}

\theoremstyle{remark}

% Marks the beginning of the document
\begin{document}
\bibliographystyle{IEEEtran}
\vspace{3cm}

\title{2013-AE-"14-26"}
\author{ai24btech11028 - Ronit Ranjan}
\maketitle
\bigskip


\begin{enumerate}
    \item The critical Mach number for a flat plate of zero thickness, at zero angle of attack, is \underline{\hspace{2cm}}

    \item A damped single degree-of-freedom system is vibrating under a harmonic excitation with an amplitude ratio of 2.5 at resonance. The damping ratio of the system is \underline{\hspace{2cm}}

    \item The cross-section of a long thin-walled member is as shown in the figure. When subjected to pure twist, point A 
    
    \begin{figure}[!ht]
    \centering
    \resizebox{0.4\textwidth}{!}{%
    \begin{circuitikz}
    \tikzstyle{every node}=[font=\large]
    \draw (1.5,12.5) to[short] (1.5,10.5);
    \draw (1.5,10.5) to[short] (7,10.5);
    \node [font=\large] at (1.25,10.5) {A};
    \end{circuitikz}
    }%
    \label{fig:my_label}
    \end{figure}
    \begin{enumerate}
        \item does not move horizontally or axially, but moves vertically 
        \item does not move axially, but moves both vertically and horizontally
        \item does not move horizontally, vertically or axially
        \item does not move vertically or axially, but moves horizontally
    \end{enumerate} 

    \item The channel section of uniform thickness 2mm shown in the figure is subjected to a torque of 10 Nm. If it is made of a material with shear modulus of 25 GPa, the twist per unit length in radians/m is \underline{\hspace{2cm}}
    \begin{figure}[!ht]
    \centering
    \resizebox{0.2\textwidth}{!}{%
    \begin{circuitikz}
    \tikzstyle{every node}=[font=\large]
    \draw (2.75,12.25) to[short] (5.5,12.25);
    \draw (2.75,12.25) to[short] (2.75,7.5);
    \draw (2.75,7.5) to[short] (5.5,7.5);
    \node [font=\large] at (4,12.5) {300mm};
    \node [font=\large] at (2,10) {600mm};
    \node [font=\large] at (4,7.25) {300mm};
    \end{circuitikz}
    }%

    \label{fig:my_label}
    \end{figure}

    \item The stiffened cross-section of a long slender uniform structural member is idealized as shown in the figure below. The lumped areas at A, B, C and D have equal cross-sectional area of 3 cm$^2$. The webs AB, BC, CD and DA are each 5 mm thick. The structural member is subjected to a twisting moment of 10 kNm. The magnitudes of the shear flow in the webs, $q_{AB}, q_{BC}, q_{CD}, and q_{DA}$ in kN/m are, respectively
    \begin{figure}[!ht]
\centering
\resizebox{0.25\textwidth}{!}{%
\begin{circuitikz}
\tikzstyle{every node}=[font=\large]
\draw (1.75,12) to[short] (5.75,12);
\draw  (1.5,12) circle (0.25cm);
\draw  (6,12) circle (0.25cm);
\draw  (1.5,9.25) circle (0.25cm);
\draw  (6,9.25) circle (0.25cm);
\draw (1.5,11.75) to[short] (1.5,9.5);
\draw (1.75,9.25) to[short] (5.75,9.25);
\draw (6,11.75) to[short] (6,9.5);
\node [font=\large] at (1,12) {A};
\node [font=\large] at (1,9.25) {B};
\node [font=\large] at (6.5,9.25) {C};
\node [font=\large] at (6.5,12) {D};
\draw [<->, >=Stealth] (7,12) -- (7,9.25);
\draw [<->, >=Stealth] (1.5,8.75) -- (6,8.75);
\node [font=\large] at (3.75,8.25) {500mm};
\node [font=\large] at (8,10.75) {200mm};
\end{circuitikz}
}%

\label{fig:my_label}
\end{figure}

    \begin{enumerate}
        \item $20, 20, 20, 20$
        \item $0, 0, 50, 50$
        \item $40, 40, 0, 0$
        \item $50, 50, 50, 50$
    \end{enumerate}

    \item Consider two engines P and Q. In P, the high pressure turbine blades are cooled with a bleed of 5\% from the compressor after the compression process and in Q the turbine blades are not cooled. Comparing engine P with engine Q, which one of the following is NOT TRUE? 
    \begin{enumerate}
        \item Turbine inlet temperature is higher for engine P 
        \item Specific thrust is higher for engine P 
        \item Compressor work is the same for both P and Q
        \item Fuel flow rate is lower for engine P 
    \end{enumerate}

    \item The mass flow rate of air through an aircraft engine is 10 kg/s. The compressor outlet temperature is 400 K and the turbine inlet temperature is 1800 K. The heating value of the fuel is 42 MJ/kg and the specific heat at constant pressure is 1 kJ/kg-K. The mass flow rate of the fuel in kg/s is approximately \underline{\hspace{2cm}}

    \item For a given inlet condition, if the turbine inlet temperature is fixed, what value of compressor efficiency given below leads to the lowest amount of fuel added in the combustor of a gas turbine engine?
    \begin{enumerate}
        \item $1$
        \item $0.95$
        \item $0.85$
        \item $0.8$
    \end{enumerate}

    \item A gas turbine engine is mounted on an aircraft which can attain a maximum altitude of 11 km from sea level. The combustor volume of this engine is decided based on conditions at 
    \begin{enumerate}
        \item sea level
        \item 8 km altitude
        \item 5.5 km altitude
        \item 11 km altitude
    \end{enumerate}

    \item Consider the low earth orbit (LEO) and the geo synchronous orbit (GSO). Then 

    \begin{enumerate}
        \item $\Delta V$ requirement for launch to LEO is greater than that for GSO, and altitude of LEO is lower than that of GSO 
        \item $\Delta V$ requirement for launch to LEO is lower than that for GSO, and altitude of LEO is lower than that of GSO         
        \item $\Delta V$ requirement for launch to LEO is greater than that for GSO, and altitude of LEO is greater than that of GSO 
        \item $\Delta V$ requirement for launch to LEO is lower than that for GSO, and altitude of LEO is greater than that of GSO         
    \end{enumerate}

    \item Which one of the following shows the CORRECT variation of stagnation temperature along the axis of an ideal ram jet engine?
    \begin{enumerate}
        \item 
        \begin{minipage}{0.3\textwidth}
            \resizebox{0.9\textwidth}{!}{%
                \begin{circuitikz}
                    \tikzstyle{every node}=[font=\small]
                    \draw [->, >=Stealth] (2,8.75) -- (6.75,8.75);
                    \draw [->, >=Stealth] (2,8.75) -- (2,13);
                    \draw [dashed] (3,12.5) -- (3,8.5);
                    \draw [dashed] (4.5,12.5) -- (4.5,8.75);
                    \draw [dashed] (5.75,12.5) -- (5.75,8.75);
                    \draw [short] (2,9.5) -- (3,10);
                    \draw [short] (3,10) -- (4.5,11.75);
                    \draw [short] (4.5,11.75) -- (5.75,10.75);
                    \draw [line width=0.2pt, short] (2,9.5) -- (5.75,9.5);
                    \node [font=\small, rotate around={90:(0,0)}] at (1.75,10.75) {Stagnation Temp.};
                    \node [font=\small] at (4,8.5) {Axis};
                    \node [font=\footnotesize] at (2.5,9.25) {Intake};
                    \node [font=\footnotesize] at (3.75,9.25) {Combustor};
                    \node [font=\footnotesize] at (5.1,9.25) {Nozzle};
                \end{circuitikz}
            }%
        \end{minipage}

        \item
        \begin{minipage}{0.3\textwidth}
            \resizebox{0.9\textwidth}{!}{%
                \begin{circuitikz}
\tikzstyle{every node}=[font=\small]
\draw [->, >=Stealth] (2,8.75) -- (6.75,8.75);
\draw [->, >=Stealth] (2,8.75) -- (2,13);
\draw [dashed] (3,12.5) -- (3,8.5);
\draw [dashed] (4.5,12.5) -- (4.5,8.75);
\draw [dashed] (5.75,12.5) -- (5.75,8.75);
\draw [line width=0.2pt, short] (2,9.5) -- (5.75,9.5);
\node [font=\small, rotate around={90:(0,0)}] at (1.75,10.75) {Stagnation Temp.};
\node [font=\small] at (4,8.5) {Axis};
\node [font=\footnotesize] at (2.5,9.25) {Intake};
\node [font=\footnotesize] at (3.75,9.25) {Combustor};
\node [font=\footnotesize] at (5,9.25) {Nozzle};
\draw [ line width=0.2pt](2,9.5) to[short] (3,9.5);
\draw [ line width=0.5pt](2,9.5) to[short] (3,9.5);
\draw [line width=0.5pt, short] (3,9.5) -- (4.5,11.75);
\draw [line width=0.5pt, short] (4.5,11.75) -- (5.75,11.75);

\label{fig:my_label}

                \end{circuitikz}
            }%
        \end{minipage} 

        \item 
        \begin{minipage}[t]{\textwidth} 
    \resizebox{0.3\textwidth}{!}{%
        \begin{circuitikz}
            \tikzstyle{every node}=[font=\small]
            \draw [->, >=Stealth] (2,8.75) -- (6.75,8.75);
            \draw [->, >=Stealth] (2,8.75) -- (2,13);
            \draw [dashed] (3,12.5) -- (3,8.5);
            \draw [dashed] (4.5,12.5) -- (4.5,8.75);
            \draw [dashed] (5.75,12.5) -- (5.75,8.75);
            \draw [line width=0.2pt, short] (2,9.5) -- (5.75,9.5);
            \node [font=\small, rotate around={90:(0,0)}] at (1.75,10.75) {Stagnation Temp.};
            \node [font=\small] at (4,8.5) {Axis};
            \node [font=\footnotesize] at (2.5,9.25) {Intake};
            \node [font=\footnotesize] at (3.75,9.25) {Combustor};
            \node [font=\footnotesize] at (5,9.25) {Nozzle};
            \draw [line width=0.2pt] (2,9.5) to[short] (3,9.5);
            \draw [line width=0.5pt, short] (2,9.5) -- (3,10);
            \draw [line width=0.5pt, short] (3,10) -- (4.5,10);
            \draw [line width=0.5pt, short] (4.5,10) -- (5.75,9.75);
        \end{circuitikz}
    }
\end{minipage}
\item

        \resizebox{0.3\textwidth}{!}{%
    \begin{circuitikz}
        \tikzstyle{every node}=[font=\small]
        \draw [->, >=Stealth] (2,8.75) -- (6.75,8.75);
        \draw [->, >=Stealth] (2,8.75) -- (2,13);
        \draw [dashed] (3,12.5) -- (3,8.5);
        \draw [dashed] (4.5,12.5) -- (4.5,8.75);
        \draw [dashed] (5.75,12.5) -- (5.75,8.75);
        \draw [line width=0.2pt, short] (2,9.5) -- (5.75,9.5);
        \node [font=\normalsize, rotate around={90:(0,0)}] at (1.75,10.75) {Stagnation Temp.};
        \node [font=\normalsize] at (4,8.5) {Axis};
        \node [font=\small] at (2.5,9.25) {Intake};
        \node [font=\small] at (3.75,9.25) {Combustor};
        \node [font=\small] at (5,9.25) {Nozzle};
        \draw [line width=0.2pt] (2,9.5) to[short] (3,9.5);
        \draw [line width=0.5pt, short] (2,9.5) -- (3,10);
        \draw [line width=0.5pt, short] (3,10) -- (4.5,11.75);
        \draw [line width=0.5pt, short] (4.5,11.75) -- (5.75,11.75);
    \end{circuitikz}
} 
        
    \end{enumerate}


    \item A rocket motor has a chamber pressure of 100 bar and chamber temperature of 3000 K. The ambient pressure is 1 bar. Assume that the specific heat at constant pressure is 1 kJ/kg-K. Also assume that the flow in the nozzle is isentropic and optimally expanded. The exit static temperature in K is 

    \begin{enumerate}
        \item $805$
        \item $845$
        \item $905$
        \item $945$
    \end{enumerate}

    \subsection*{Q.26 to Q.55 carry two marks each}

    \item I = $\iint_S \brak{ y^2 \hat{i} + z^2 \hat{j} + x^2 \hat{k} }\brak{ \cdot ( x \hat{i} + y \hat{j} + z \hat{k}} \, dS,$  where S denotes the surface of the sphere of unit radius centered at the origin. Here $\hat{i}, \hat{j}$ and $\hat{k}$  denote three orthogonal unit vectors. The value of I is \underline{\hspace{2cm}}



    
\end{enumerate}



\end{document}
